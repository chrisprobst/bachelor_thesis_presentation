%!TEX root = master.tex

\section{Einleitung}

\subsection{Problemstellung}
\subsection{Motivation}
\subsection{Zielsetzung}

\begin{frame}
  \frametitle{Problemstellung}
	
  \setbeamertemplate{itemize items}[circle]
  \begin{itemize}
	  \item Möchte man Daten von einem einzelnen Computer aus zu beliebig vielen anderen Computern transferieren, so wird häufig ein Client\,/\,Server Netzwerk genutzt. 
	  \item Diese Methode ist einfach, zuverlässig und lang erprobt. Jedoch gibt es Skalierungsprobleme:

 	  \setbeamertemplate{itemize items}[default]
	  \begin{itemize}
	    \item Jeder Client erhöht die Last des Servers
	    \item Mehr Clienten benötigen mehr Server und mehr Bandbreite
	    \item Uploadbandbreite der Clienten bleibt genutzt
	 
	  \end{itemize}
  \end{itemize}
\end{frame}


\begin{frame}
  \frametitle{Motivation}

  \setbeamertemplate{itemize items}[circle]
  \begin{itemize}
	  \item Um ohne ein skalierbares Serversystem Daten dennoch schnell zu verbreiten, kann ein Peer-to-Peer Netzwerk benutzt werden, wo jeder Computer einen Peer darstellt und bei der Datenverbreitung hilft. Einige Anwendungsfälle sind:

	  \setbeamertemplate{itemize items}[default]
	  \begin{itemize}
	    \item File Sharing (BitTorrent)
	    \item Audio und Video Streaming (Skype)    
	    \item DHTs (Kademlia, Chord)
	  \end{itemize}

	  \setbeamertemplate{itemize items}[circle]
	  \item Geräte mit einer geringen Uploadbandbreite können so dennoch schnell Daten verbreiten, wie z.B. ein Live-Videostream von einem Smartphone aus.
  \end{itemize}
\end{frame}


\begin{frame}
  \frametitle{Zielsetzung}

      Implementierung einer Peer-to-Peer Anwendung, die Daten ausgehend von einem Peer, auch genannt Super-Peer, an beliebig viele Peers versendet. Dabei sollen folgende Bedingungen eingehalten werden:

	  \begin{itemize}
	    \item Möglichst hohe Auslastung der Uploadbandbreite der einzelnen Peers.
	    \item Jeder Peer soll die Daten möglichst zur gleichen Zeit fertigstellen.
	    \item Gesamtdauer unabhängig von der Anzahl der Peers.
	    \item Gesamtdauer kleiner als $2 * T_0$.
	  \end{itemize}

\end{frame}



% \begin{frame}
%   \frametitle{Motivation}

%   Sehr praktisch ist die Verwendung von Blöcken:
%   \begin{block}{Normaler Blocktitel}
%     Blöcke sind zur Hervorhebung gedacht. In normalen Blöcken können
%     wichtige Erkenntnisse (Zwischenergebnisse) stehen, die nicht
%     unbemerkt bleiben dürfen.
%   \end{block}

%   \begin{exampleblock}{Beispiel-Blocktitel}
%     Diese Blöcke sind für Beispiele o.ä. gedacht.
%   \end{exampleblock}

%   \begin{alertblock}{Alarm-Blocktitel}
%     Diese Blöcke beinhalten für gewöhnlich Problembeschreibungen.
%   \end{alertblock}
% \end{frame}

% \begin{frame}
%   \frametitle{Weitere Hilfen}

%   Es gibt viele Quellen, die für \LaTeX Beamer herangezogen werden können. Besonders gut sind:
%   \begin{itemize}
%     \item Der Beamer User Guide von Till Tantau:\\\texttt{http://www.math.binghamton.edu/erik/beameruserguide.pdf}
%     \item Ein sehr gutes Beamer Tutorial von Ki-Joo Kim:\\\texttt{http://saikat.guha.cc/ref/beamer\_guide.pdf}
%     \item Der jeweilige Betreuer der Abschlussarbeit :-)
%   \end{itemize}

% \end{frame}
