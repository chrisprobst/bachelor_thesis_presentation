\section{Einleitung}

\begin{frame}
  \frametitle{Motivation}  
  \begin{itemize}
    \item Problem: Datenverbreitung mit Client\,/\,Server Architektur skaliert nicht
    \vspace{1mm}
    \begin{itemize}
      \item Jeder Client erhöht die Auslastung
      \item Uploadbandbreite des Servers ist limitiert        
      \item Uploadbandbreite der Clients ungenutzt
    \end{itemize}

    \vspace{2mm}

    \item Lösungsansatz: Verwendung eines Peer-to-Peer Netzwerks
    \vspace{1mm}
    \begin{itemize}
      \item Jeder Peer hilft bei der Datenverbreitung
      \item Der Super-Peer (Peer mit vollständigem Datensatz) wird nicht stärker ausgelastet als andere Peers
      \item Ein Super-Peer mit geringer Uploadbandbreite kann dennoch schnell Daten verbreiten
    \end{itemize}    
  \end{itemize}
\end{frame}


\begin{frame}
  \frametitle{Zielsetzung}
    Implementierung einer Peer-to-Peer Anwendung zur termingerechten Verbreitung von Daten:
    \vspace{1mm}
    \begin{itemize}
      \item Uploadbandbreite der Peers effizient nutzen
      \item Zeitpunkt für den Erhalt der Daten bei allen Peers gleich
      \item Gesamtdauer unabhängig von der Anzahl der Peers
      \item Gesamtdauer kleiner als $2 * T_0$.
    \end{itemize}
\end{frame}